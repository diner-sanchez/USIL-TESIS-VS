\usepackage{lipsum} % Generar texto ficticio Lorem Ipsum
\usepackage[american]{babel} % Configuración del idioma en español
\usepackage[utf8]{inputenc}
\usepackage[T1]{fontenc}
\usepackage{csquotes} % Mejora las citas y referencias bibliográficas
\usepackage[backend=biber, style=apa]{biblatex} % Estilo APA para citas y referencias bibliográficas
\addbibresource{20_Bibliography/bibliography.bib} 
\usepackage{hyperref} % Creación de enlaces internos y externos en el documento PDF
\hypersetup{
    colorlinks=true, % Activar enlaces coloreados
    linkcolor=black, % Color de los enlaces internos (por defecto: red)
    citecolor=black, % Color de los enlaces de citas (por defecto: green)
    urlcolor=black, % Color de los enlaces URL (por defecto: magenta)
    linkbordercolor=white, % Color del borde de los enlaces internos (por defecto: red)
    citebordercolor=white, % Color del borde de los enlaces de citas (por defecto: green)
    urlbordercolor=white % Color del borde de los enlaces URL (por defecto: magenta)
}
\usepackage{subcaption} % Permite crear subfiguras
\usepackage{mathptmx} % Cargar la fuente Times New Roman


%%%%%%%%% INCLUIR Y NUMERAR SECCIONES EN TOC Y DOCUMENTO %%%%%%%%%%%%%

\setcounter{tocdepth}{2} % Incluir hasta nivel de subsecciones en la tabla de contenidos
\setcounter{secnumdepth}{3} % Numerar hasta nivel de subsecciones
\AtBeginDocument{\addtocontents{toc}{\protect\thispagestyle{empty}}} % Desactivar numeración de página en la página de la tabla de contenidos

%%%%%%%%% AGREGAR PREFIJO A LOF Y LOT %%%%%%%%%%%%%

\renewcommand{\cftfigpresnum}{Figura }
\setlength{\cftfignumwidth}{5em}

\renewcommand{\cfttabpresnum}{Tabla }
\setlength{\cfttabnumwidth}{5em}

\renewcommand{\cftmyequationspresnum}{Ecuación }
\setlength{\cftmyequationsnumwidth}{6em}

%%%%%%%%% MODIFICAR ENTORNO SUBFIGURE %%%%%%%%%%%%%

\renewcommand\thesubfigure{\Alph{subfigure}}
\captionsetup[subfigure]{labelformat=empty}

%%%%%%%%%% Centrado del título del ÍNDICE / LISTA DE FIGURAS / LISTA DE CUADROS %%%%%%%%%

\renewcommand{\cfttoctitlefont}{\hfill \normalfont\normalsize\bfseries}
\renewcommand{\cftaftertoctitle}{\hfill}

\renewcommand{\cftlottitlefont}{\hfill\normalfont\normalsize\bfseries}
\renewcommand{\cftafterlottitle}{\hfill}

\renewcommand{\cftloftitlefont}{\hfill\normalfont\normalsize\bfseries}
\renewcommand{\cftafterloftitle}{\hfill}

\renewcommand{\listequationsname}{\hfill\normalfont\normalsize\bfseries Lista de Ecuaciones \hfill}

\renewcommand{\listofabbreviationsname}{\hfill\normalfont\normalsize\bfseries Lista de Siglas \hfill}

\renewcommand{\listofsymbolsname}{\hfill\normalfont\normalsize\bfseries Lista de Simbolos \hfill}

%%%%%%%%%% AGREGAR PREFIJO DE A SECCIONES Y A TOC %%%%%%%%%

\makeatletter
% Agregar prefijo a las secciones en la tabla de contenidos
%\renewcommand{\cftsecpresnum}{Capítulo }
%\renewcommand{\cftsecaftersnum}{:}
\settowidth{\cftsecnumwidth}{\normalfont\bfseries\thesection\hspace{1em}} % Calcula el ancho necesario
\addtolength{\cftsecnumwidth}{1.0em} % Agrega un espaciado adicional
\makeatother

\setlength{\cftbeforesecskip}{2mm}
\renewcommand\cftsecafterpnum{\vskip5pt}
\renewcommand\cftsubsecafterpnum{\vskip5pt}
\renewcommand\cftsubsubsecafterpnum{\vskip5pt}

%%%%%%%%%%%%%% DEFINIR INFORMACIÓN PERSONAL Y TRABAJO %%%%%%%%%%%%%%%%%%%%

%Título de la DOCUMENTO (Siempre en mayuscula y sin saltos de linea)
\title{PROPUESTA DE IMPLEMENTACIÓN DE UN REFORZAMIENTO ESTRUCTURAL PARA REDUCIR LA VULNERABILIDAD SÍSMICA DE LA IGLESIA VILLA DE HUAYLLAY USANDO MODELO DE ELEMENTOS FINITOS}

%Título corto del DOCUMENTO al pie de PÁGINA (Agregar salto de línea de ser necesario)
\shorttitle{PROPUESTA DE IMPLEMENTACIÓN DE UN REFORZAMIENTO ESTRUCTURAL PARA REDUCIR LA VULNERABILIDAD SÍSMICA DE LA IGLESIA VILLA DE HUAYLLAY USANDO MODELO DE ELEMENTOS FINITOS}

%Nombre de la FACULTAD (Siempre en mayuscula)
\faculty{FACULTAD DE INGENIERÍA}

%Para obtener el GRADO profesional de ...
\degree{Ingeniería Civil}

%AUTOR para carátula (Siempre en mayuscula y sin saltos de linea)
\authorname{SANCHEZ CARBAJAL, DINER ERICK}
\authororcid{(XXXX-XXXX-XXXX-XXXX)}

%ASESOR para carátula (Siempre en mayuscula y sin saltos de linea)
\advisorname{ING. SOTO OBLEA, EDWARD JONATHAN}
\advisororcid{(XXXX-XXXX-XXXX-0000)}

%AÑO para carátula
\yyearr{2023}

%%%%%%%%%%%%%%%%% DEFINIR LOS META DATOS del PDF %%%%%%%%%%%%%%%%%%%%%%%%%

\makeatletter
\hypersetup{
  pdftitle={\@title},
  pdfauthor={\@authorname},
  pdfsubject={Tesis de grado},
  pdfkeywords={word1,word2,word3,word4,word5},
}
\makeatother

%%%%%%%%%%%%%%%%%%%%%%%%%%%%%%%%%%%%%%%%%%%%%%%%%%%%%%%%%%%%%%%%%%%%